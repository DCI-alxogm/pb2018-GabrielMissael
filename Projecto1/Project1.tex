\documentclass[a4paper,10pt]{article}
%\documentclass[a4paper,10pt]{scrartcl}

\usepackage[utf8x]{inputenc}
\usepackage{graphicx}
\usepackage{textcomp}
\usepackage{amsmath}



\title{\vspace{-5cm} Programación básica\\ 
Proyecto 1: \'Orbita de un planeta  entorno a una estrella.}

\author{Prof. Alma González}

\pdfinfo{%
  /Title    ()
  /Author   ()
  /Creator  ()
  /Producer ()
  /Subject  ()
  /Keywords ()
}

\begin{document}
\maketitle
\begin{enumerate}

\item 
La dinámica de un planeta moviendose alrededor de una estrella puede obtenerse a partir de resolver la ecuación :

\begin{equation}
\vec{F}=m_p\,\vec{a_p}.
\label{eq:F1}
\end{equation}

La fuerza que siente el planeta  es la fuerza de atracción gravitacional debido a la presencia de la estrella, dada por: 
\begin{equation}
\vec{F}=-\frac{G M_{*} m_p}{r^3}\vec{r},
\label{eq:F2}
\end{equation}

donde $G=4\pi^2 yr^{-2} AU M_{\odot}^{-1}$ es la constante de Gravitación Universal, en unidades de Masas Solares, unidad astronomica y años.  Combinando las ecuaciones  \ref{eq:F1} y \ref{eq:F2} se obtiene la ecuación de movimiento:

\begin{equation}
m_p \ddot{\vec{r}}=- \frac{G M_{*} m_{p} \vec{r}}{r^3}.
\label{eq:mov}
\end{equation}

Nota que $\vec{r}$ es el vector distancia entre la estrella y el planeta y $r$ es su magnitud.  Dado que la ecuación \ref{eq:mov} es vectorial, podemos separarla en  componentes (usaremos coordenadas cartesianas centradas en la estrella) :

\begin{eqnarray}
 \ddot{x}=-\frac{G M_{*} x}{r^3},\\
  \ddot{y}=-\frac{G M_{*} y}{r^3},\\
 \ddot{z}=-\frac{G M_{*} z}{r^3},
\end{eqnarray}

y $r=\sqrt{x^2+y^2+z^2}$. Las ecuaciones diferenciales anteriores, de segundo orden, pueden reescribirse de la forma :

\begin{eqnarray}
 \dot{v_x}=-\frac{G M_{*} x}{r^3}\\
 \dot{x}=v_x\\
  \dot{v_y}=-\frac{G M_{*} y}{r^3}\\
  \dot{y}=v_y\\
 \dot{v_z}=-\frac{G M_{*} z}{r^3}\\
  \dot{z}=v_z
\end{eqnarray}

Entonces, resolver la \'orbita del planeta significa  resolver las ecuaciones anteriores, para encontrar $x(t), y(t), z(t)$; en el proceso también se calcula  $v_x(t), v_y(t), v_z(t)$. Es decir se calculan las posiciones y las velocidades del planeta a cada paso de tiempo. 

 El método más sencillo de implementar es el de Euler. Consiste en establecer un conjunto de posiciones ($x_0, y_0, z_0$) y velocidades iniciales ($v_{x_0}, v_{y_0}, v_{z_0}$) a un tiempo inicial $t_0$, y actualizarlas a un tiempo  posterior  $t_{i}$ para obtener las nuevas posiciones y velocidades en el tiempo i-esimo, siguiendo la siguiente regla :

\begin{eqnarray}
x_i &=&x_0+v_{x_0}*h\\
y_i &=&x_0+v_{y_0}*h\\
z_0 &=&x_0+v_{z_0}*h\\
v_{x,i} &=&v_{x,0}-h*\frac{G M_{*} x_0}{r_{t_0}^3}\\
v_{y,i} &=&v_{y,0}-h*\frac{G M_{*} y_0}{r_{t_0}^3}\\
v_{z,i} &=&v_{z,0}-h*\frac{G M_{*} z_0}{r_{t_0}^3}\\
%v_x(t_{i+1})=v_x(t_i)-h*\frac{G M_{*} x(t_i)}{r(t_i)^3}\\
%
%y(t_{i+1})=y(t_i)+v_y(t_i)*h\\
%v_y(t_{i+1})=v_y(t_i)-h*\frac{G M_{*} y(t_i)}{r(t_i)^3}\\
%z(t_{i+1})=z(t_i)+v_z(t_i)*h\\
%v_z(t_{i+1})=v_z(t_i)-h*\frac{G M_{*} z(t_i)}{r(t_i)^3}
\end{eqnarray}

donde $h$ es el paso de tiempo $h=t_{i}-t_{0}$.  A cada paso de tiempo se remplazan las posiciones y velocidades iniciales por las nuevas, para poder dar un nuevo paso de tiempo. 

Esto es: las coordenadas y las velocidades , al tiempo$ t_{i}$ dependen de las velocidades y las posiciones al tiempo $t_{0}$. Consideraremos un paso de tiempo constante y daremos un valor suficientemente pequeño para que la evolución de la órbita sea precisa (h~0.001 cuando la velocidad está dada en AU/yr ). Nota que h es un par\'ametro con el que se deber\'an hacer pruebas, hasta obtener un buen resultado.

{\bf{Proyecto}}:

Se usar\'a la infomaci\'on anterior para calcular la \'orbita de los planetas de nuestro sistema solar entorno al sol. Por ejemplo: en unidades usuales en astronomía, la masa del sol es $M_{sol}$=1$M_{\odot}$, mientras que la masa de los planeta es una fracción  de \'esta, por ejemplo la masa de la tierra es  $M_{T}=3\times10^{-6} M_{\odot}$.  La distancia de la tierra al sol es de $d=1AU$. 

Para completar este proyecto se requiere :

\begin{enumerate}
 \item Hacer el algoritmo o diagrama de flujo del  programa a realizar, i.e., explicar la secuencia de pasos a seguir en el programa. {\bf (1 Punto)}.
 
 \item El programa debe {\bf(5 puntos)}:
 \begin{itemize}
	\item Leer a partir de un archivo los parámetros (masa de los planetas, masa de la estrella, el tiempo total de la evolución (recomendado que sean multiplos de años), y el tama\~no del incremento temporal (h, se da en las mismas unidades que el tiempo total, y suele ser una fracción pequeña del mismo.). Del mismo archivo tambi\'en se deben leer  6 columnas con las posiciones y velocidades iniciales del planeta (ver tabla)(x,y,z,vx,vy,vz). (OJO: En este proyecto solo est\'amos resolviendo la orbita de 1 planeta a la vez, no de todos de forma auto-consistente) 
	
\item Se debe hacer la evolución temporal del método de Euler, para el n\'umero de pasos de tiempo definidos por las variables tiempo total de evolución y tamaño del incremento temporal. Todas las variables deben ser definidas adecuadamente al tipo de cantidad que representan.   

\item Verificar que todas las variables se hayan definido adecuadamente al tipo de variables que son, i.e., definir aquellas que son enteros como enteros, etc...   
 \end{itemize}

\item Guardar en un archivo las nuevas posiciones y velocidades de cada planeta a cada paso de la  evolución.{\bf(1Punto)}. 

\item Gráficar la órbita obtenida ($x Vs y$, $x Vs z$, $y Vs z$), pueden usar el graficador de su preferencia  {\bf(1 Punto)}.
\item Se evaluar\'a la presentaci\'on de los resultados. Que el programa est\'e debidamente comentado, compile y se ejecute correctamente{\bf(1 Punto)}. 

\item El programa debe hacerse funcionar primero para un solo planeta, y posteriormente generalizar para hacer todos los planetas a la vez. En este caso, se puede guardar la trayectoria de cada planeta en un archivo independiente para cada uno. {\bf(1 Punto)}

\end{enumerate}

 Extra (1 punto): En la gráfica de la \'orbita, sobre los puntos poner una línea que corresponda a la solución an\'alitica. 






\end{enumerate}

\end{document}

